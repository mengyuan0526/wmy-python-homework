\documentclass[UTF8,a4paper,12pt]{ctexart}  % Latex 去掉上面的语句,加上本语句
\usepackage{xeCJK}
\setCJKmainfont[BoldFont=AdobeHeitiStd-Regular]{AdobeSongStd-Light}
\setCJKfamilyfont{song}{AdobeSongStd-Light}
\setCJKfamilyfont{hei}{AdobeHeitiStd-Regular}
\setCJKfamilyfont{kai}{AdobeKaitiStd-Regular}
\setCJKfamilyfont{fs}{Sun Yat-sen Hsingshu}
% \setCJKmainfont[BoldFont=SimHei]{SimSun}


\renewcommand{\contentsname}{\centerline{\textcolor{violet}{目 \ \ 录}}}    % 将Contents改为目录
\renewcommand{\abstractname}{摘 \ \ 要}      % 将Abstract改为摘要
\renewcommand{\refname}{参考文献}            % 将Reference改为参考文献
\renewcommand\tablename{表}
\renewcommand\figurename{图}
\renewcommand{\today}{\number\year 年 \number\month 月 \number\day 日}

\usepackage[dvipsnames]{xcolor}
\PassOptionsToPackage{colorlinks=true,citecolor=blue, urlcolor=blue, linkcolor=violet, bookmarksdepth=4}{hyperref}

\usepackage{lscape}
\usepackage{indentfirst}
\usepackage{textcomp}                      % provide many text symbols
\usepackage{setspace}                      % 各种间距设置

% ---------------------------------Table------------------------------
\usepackage{longtable}
\usepackage{booktabs}
\usepackage{array}                         % 提供表格中每一列的宽度及位置支持
\usepackage{rotating}
\usepackage{multirow}
\usepackage{wrapfig}
\usepackage{colortbl}
\usepackage{pdflscape}
\usepackage{tabu}
\usepackage{threeparttable}
\usepackage{threeparttablex}
\usepackage[normalem]{ulem}
\usepackage{makecell}
\newcolumntype{L}[1]{>{\raggedright\let\newline\\\arraybackslash\hspace{0pt}}m{#1}}
\newcolumntype{C}[1]{>{\centering\let\newline\\\arraybackslash\hspace{0pt}}m{#1}}
\newcolumntype{R}[1]{>{\raggedleft\let\newline\\\arraybackslash\hspace{0pt}}m{#1}}

%% 参考文献
\usepackage{gbt7714}
\usepackage{natbib}
\setlength{\bibsep}{0.5pt}


\usepackage[utf8]{inputenc}
% \usepackage[T1]{fontenc} % [T1] 主要支持东欧等国家重音符 , 与下面 consolas 冲突
\usepackage{fontenc}
\usepackage{fixltx2e}
\usepackage{graphicx}
\usepackage{float}
\usepackage{wrapfig}
\usepackage{soul}
\usepackage{textcomp}

\newcommand\hmmax{0} %% 防止Too many math alphabets used in version normal.
\newcommand\bmmax{0} %% 防止Too many math alphabets used in version normal.

\usepackage{lmodern,bm}   % 必需出现在amsmath等包前面,否则会出错
\usepackage{amsmath}
\usepackage{marvosym}
\usepackage{wasysym}
\usepackage{latexsym}
\usepackage{amssymb}
\usepackage{hyperref}
\usepackage{listings}
\usepackage{tikz}

\setmonofont{Consolas} % listings 中支持 consolas 字体,必需配合上面\usepackage{fontenc} 中不出现[T1]才可以

\lstset{numbers=left, numberstyle=\ttfamily\tiny\color{Gray}, stepnumber=1, numbersep=8pt,
  frame=leftline,
  framexleftmargin=0mm,
  rulecolor=\color{CadetBlue},
  backgroundcolor=\color{Periwinkle!20},
  stringstyle=\color{CadetBlue},
  flexiblecolumns=false,
  aboveskip=5pt,
  belowskip=0pt,
  language=R,
  basicstyle=\ttfamily\footnotesize,
  columns=flexible,
  keepspaces=true,
  breaklines=true,
  extendedchars=true,
  texcl=false,  % 必须设置为false设置为true的时候 R 代码中不能含有多个注释符号 #
  upquote=true,
  showstringspaces=false,
  keywordstyle=\bfseries,
  keywordstyle=\color{Purple},
  xleftmargin=20pt,
  xrightmargin=10pt,
  morecomment=[s]{\#}{\#},
  commentstyle=\color{OliveGreen!60}\scriptsize,
  tabsize=4}

\tolerance=1000

%======================== 根据选项设置代码处理方式 RMARKDOWN 中独有=============

\providecommand{\tightlist}{\setlength{\itemsep}{0pt}\setlength{\parskip}{0pt}}
\newcommand{\passthrough}[1]{\lstset{mathescape=false}#1\lstset{mathescape=true}}

\author{\CJKfamily{kai} 金 \enspace 林 \\ \CJKfamily{kai} 中南财经政法大学统计系 \\ jinlin82@gmail.com}
% ------------------------Chapter Section Title-------------------------
%--- 英文期刊标题 -----
\usepackage{titlesec}
\titleformat{\section}{\large\bfseries}{\thesection}{1em}{}
\titleformat{\subsection}{\normalsize\bfseries}{\thesubsection}{0.5em}{}
\titlespacing{\section}{0pt}{1ex plus 1ex minus .2ex}{1ex plus 1ex minus .2ex}
\titlespacing{\subsection}{0pt}{0.5ex plus 1ex minus .2ex}{0.5ex plus 1ex minus .2ex}
%--- 中文期刊标题 -----
% \CTEXsetup[name={,、}, number={\chinese{section}}, aftername={},
% format={\large \heiti }, indent={24pt},
% beforeskip={1ex plus 1ex minus .2ex},
% afterskip={1ex plus 1ex minus .2ex}]
% {section}
% \CTEXsetup[name={(,)}, number={\chinese{subsection}}, aftername={},
% format={\normalsize \bfseries \songti}, indent={\parindent},
% beforeskip={0.5ex plus 1ex minus .2ex},
% afterskip={0.5ex plus 1ex minus .2ex}]
% {subsection}
% \CTEXsetup[name={,.}, number={\arabic{subsubsection}},
% aftername={}, format={\normalsize \bfseries \songti},indent={\parindent},
% beforeskip={0ex plus 1ex minus .2ex},
% afterskip={0.2ex plus 1ex minus .2ex}]
% {subsubsection}

% ------------------------Figure and Table Caption---------------------
%\makeatletter                        % 图表标题格式设置
%\renewcommand{\fnum@table}[1]{\small \bfseries\textcolor{Violet}{\tablename\thetable~~}}
%\renewcommand{\fnum@figure}[1]{\small \CJKfamily{hei} \textcolor{Violet}{\figurename\thefigure~~}}
%\makeatother

\usepackage[skip=0pt, labelsep=quad, font={small, bf}, labelfont={color={Violet}}]{caption}

\renewcommand{\thefigure}{\arabic{figure}}
\renewcommand{\thetable}{\arabic{table}}
\newcommand{\HRule}{\rule{\linewidth}{0.5mm}}

\usepackage[top=2cm,bottom=2cm,left=3cm,right=3cm]{geometry}
\sloppy
\linespread{1.2}                    % 设置行距
\setlength{\parindent}{24pt}        % 段落缩进
\setlength{\parskip}{1ex plus 0.5ex minus 0.2ex}
\pagestyle {plain}                  % 去掉页眉
\setcounter{secnumdepth}{4}

%%% Change title format to be more compact
\usepackage{titling}

% Create subtitle command for use in maketitle
\newcommand{\subtitle}[1]{
  \posttitle{
    \begin{center}\large#1\end{center}
    }
}

\setlength{\droptitle}{-2em}
  \title{\LARGE\textbf{矩阵计算python笔记}}
  \pretitle{\vspace{\droptitle}\centering\huge}
  \posttitle{\par}
  \author{王梦圆}
  \preauthor{\centering\large\emph}
  \postauthor{\par}
  \predate{\centering\large\emph}
  \postdate{\par}
  \date{2020-02}



\begin{document}
\maketitle

\section{创建一个向量}

在python中可以用numpy.array()来创建一个向量,例如:

\begin{lstlisting}[language=Python]
import numpy as np
x = np.array([1,2,3,4])
print(x)
\end{lstlisting}

\section{创建一个矩阵}

在python里面使用numpy创建矩阵有两种方法:第一种:直接法------np.matrix()创建矩阵

\begin{lstlisting}[language=Python]
x = np.matrix([[1,2,3,4],[5,6,7,8],[9,10,11,12]])
x
\end{lstlisting}

第二种:利用np.arange().reshape(shape,order)创建矩阵

\begin{lstlisting}[language=Python]
x1 = np.arange(12).reshape(3,4)#3行4列矩阵,默认按行排列
x2 = np.arange(12).reshape(4,3)#4行3列矩阵,按行排列
x3 = np.arange(12).reshape((4,3),order='F')#按列排列
\end{lstlisting}

二维数组不能直接定义行名和列名,可以先转化成DataFrame形式,再利用index和columns定义行名和列名

\begin{lstlisting}[language=Python]
import pandas as pd
index=['r1','r2','r3']
columns=['c1','c2','c3','c4']
x1 = pd.DataFrame(x1,index=index,columns=columns)
x1
\end{lstlisting}

\section{矩阵转置}

A为 \emph{m×n} 矩阵,求A的转置在python中可用A.T,例如:

\begin{lstlisting}[language=Python]
A = np.arange(1,13).reshape(3,4)
A
\end{lstlisting}

\begin{lstlisting}[language=Python]
A.T
\end{lstlisting}

在python里面若想要得到一个行向量,例如:

\begin{lstlisting}[language=Python]
x = np.array([1,2,3,4,5,6,7,8,9,10]).reshape((1,-1))
type(x)
\end{lstlisting}

利用x.T,可以得到一个列向量,例如

\begin{lstlisting}[language=Python]
x.T
\end{lstlisting}

\begin{lstlisting}[language=Python]
(x.T).shape
\end{lstlisting}

在python里面若想要得到一个列向量,例如:

\begin{lstlisting}[language=Python]
x = np.array([1,2,3,4,5,6,7,8,9,10]).reshape((-1,1))
x.shape
\end{lstlisting}

\section{矩阵相加减}

在python中对同行同列矩阵相加减,可用符号:``+''、``-'',例如:

\begin{lstlisting}[language=Python]
A = B = np.arange(1,13).reshape((3,4),order='F')
A+B
\end{lstlisting}

\begin{lstlisting}[language=Python]
A-B
\end{lstlisting}

\section{数与矩阵相乘}

\textbf{A}为 \emph{m×n} 矩阵,c\textgreater{}0,在python中求c\textbf{A}可用符号:``*'',例如:

\begin{lstlisting}[language=Python]
c =2
c*A
\end{lstlisting}

\section{矩阵相乘}

\textbf{A}为m×n矩阵,\textbf{B}为n×k矩阵,在python中求\textbf{AB}可用np.dot(),例如:

\begin{lstlisting}[language=Python]
A = np.arange(1,13).reshape((3,4),order='F')
B = np.arange(1,13).reshape((4,3),order='F')
np.dot(A,B)
\end{lstlisting}

\section{矩阵对角元素相关运算}

例如要取一个方阵的对角元素,

\begin{lstlisting}[language=Python]
A = np.arange(1,17).reshape((4,4),order='F')
np.diag(A)
\end{lstlisting}

对一个向量应用np.diag()函数将产生以这个向量为对角元素的对角矩阵,例如:

\begin{lstlisting}[language=Python]
A = np.arange(1,17).reshape((4,4),order='F')
np.diag(np.diag(A))
\end{lstlisting}

对一个正整数z应用np.identity()函数将产生以z维单位矩阵,例如:

\begin{lstlisting}[language=Python]
np.identity(3)
\end{lstlisting}

\section{矩阵求逆}

矩阵求逆可用函数np.linalg.inv(),例如:

\begin{lstlisting}[language=Python]
A = np.random.normal(0,1,(4,4))
np.linalg.inv(A)
\end{lstlisting}

\begin{lstlisting}[language=Python]
np.dot(np.linalg.inv(A),A)
\end{lstlisting}

\section{矩阵的特征值与特征向量}

矩阵 \emph{A} 的谱分解为\textbf{A=UΛU'},其中 \emph{Λ} 是由 \emph{A} 的特征值组成的对角矩阵,\emph{U} 的列为 \emph{A} 的特征值对应的特征向量,在python中可以用函数np.linalg.eig()函数得到U和Λ

\begin{lstlisting}[language=Python]
A = np.identity(4)+1
A_eig = np.linalg.eig(A)
Λ = np.diag(A_eig[0])
U = A_eig[1]
Λ,U
\end{lstlisting}

\begin{lstlisting}[language=Python]
np.dot(np.dot(U,Λ),U.T)
\end{lstlisting}

\begin{lstlisting}[language=Python]
np.dot(U.T,U)
\end{lstlisting}

\hypertarget{choleskey}{%
\section{矩阵的Choleskey分解}\label{choleskey}}

对于正定矩阵\textbf{A},可对其进行Choleskey分解,即:\textbf{A=PP'},其中P为下三角矩阵,在python中可以用函数np.linalg.cholesky()进行Choleskey分解,例如:

\begin{lstlisting}[language=Python]
A = np.identity(4)+1
cho = np.linalg.cholesky(A)
np.dot(cho,cho.T)
\end{lstlisting}

若矩阵为对称正定矩阵,可以利用Choleskey分解求行列式的值,如:

\begin{lstlisting}[language=Python]
np.prod(np.diag(cho)**2)
\end{lstlisting}

\begin{lstlisting}[language=Python]
np.linalg.det(A)
\end{lstlisting}

\section{矩阵奇异值分解}

\textbf{A}为 \emph{m×n} 矩阵,rank(A)= r, 可以分解为:\textbf{A=UDV'},其中\textbf{U'U=V'V=I}。在python中可以用函数scd()进行奇异值分解,例如:

\begin{lstlisting}[language=Python]
A = np.arange(1,19).reshape((3,6),order='F')
A
\end{lstlisting}

\begin{lstlisting}[language=Python]
d = np.linalg.svd(A,full_matrices=0)[1]
d
\end{lstlisting}

\begin{lstlisting}[language=Python]
u = np.linalg.svd(A,full_matrices=0)[0]
u
\end{lstlisting}

\begin{lstlisting}[language=Python]
v = np.linalg.svd(A,full_matrices=0)[2]
v
\end{lstlisting}

\begin{lstlisting}[language=Python]
np.dot(np.dot(u,np.diag(d)),v)
\end{lstlisting}

\begin{lstlisting}[language=Python]
np.dot(u.T,u,out=None,)
\end{lstlisting}

\begin{lstlisting}[language=Python]
np.dot(v,v.T)
\end{lstlisting}

\hypertarget{qr}{%
\section{矩阵QR分解}\label{qr}}

\textbf{A}为\emph{m×n}矩阵可以进行\emph{QR}分解,\textbf{A=QR},其中:\textbf{Q'Q=I},在python中可以用函数qr()进行QR分解,得到Q矩阵和R矩阵,例如:

\begin{lstlisting}[language=Python]
A = np.arange(1,17).reshape((4,4),order='F')
qr = np.linalg.qr(A)
Q = qr[0]
R = qr[1]
np.dot(Q,R)
\end{lstlisting}

\begin{lstlisting}[language=Python]
np.dot(Q.T,Q)
\end{lstlisting}

\hypertarget{moore-penrose}{%
\section{矩阵广义逆(Moore-Penrose)}\label{moore-penrose}}

\emph{n×m}矩阵 \textbf{A+} 称为\emph{m×n}矩阵\textbf{A}的Moore-Penrose逆,如果它满足下列条件:\textbf{①A A+A=A;②A+A A+= A+;③(A A+)H=A A+;④(A+A)H= A+A}
在python中可用np.linalg.pinv()函数,例如:

\begin{lstlisting}[language=Python]
A = np.arange(1,17).reshape((4,4),order='F')
MP = np.linalg.pinv(A)#广义逆
\end{lstlisting}

验证性质1:

\begin{lstlisting}[language=Python]
np.dot(np.dot(A,MP),A)
\end{lstlisting}

验证性质2:

\begin{lstlisting}[language=Python]
np.dot(np.dot(MP,A),MP)
\end{lstlisting}

验证性质3:

\begin{lstlisting}[language=Python]
np.dot(A,MP).T
\end{lstlisting}

\begin{lstlisting}[language=Python]
np.dot(A,MP)
\end{lstlisting}

验证性质4:

\begin{lstlisting}[language=Python]
np.dot(MP,A).T
\end{lstlisting}

\begin{lstlisting}[language=Python]
np.dot(MP,A)
\end{lstlisting}

\hypertarget{kronecker}{%
\section{矩阵Kronecker积}\label{kronecker}}

n×m矩阵A与h×k矩阵B的kronecker积为一个nh×mk维矩阵,公式为:
\[
\mathbf{A}_{m \times n} \otimes \mathbf{B}_{h \times k}=\left(\begin{array}{ccc}
a_{11} \mathbf{B} & \cdots & a_{1 n} \mathbf{B} \\
\vdots & \vdots & \vdots \\
a_{m 1} \mathbf{B} & \cdots & a_{m n} \mathbf{B}
\end{array}\right)_{m h \times nh }
\]
在python中kronecker积可以用函数np.kron()来计算,例如:

\begin{lstlisting}[language=Python]
A = np.arange(1,5).reshape((2,2),order='F')
B = np.ones((2,2))
A
\end{lstlisting}

\begin{lstlisting}[language=Python]
B
\end{lstlisting}

\begin{lstlisting}[language=Python]
np.kron(A,B)
\end{lstlisting}

\section{矩阵的维数}

在python中可以直接利用(matrix).shape得到矩阵的维数

\begin{lstlisting}[language=Python]
A = np.arange(1,13).reshape((3,4),order='F')
A
\end{lstlisting}

\begin{lstlisting}[language=Python]
A.shape
\end{lstlisting}

\section{矩阵的行和、列和、行平均与列平均}

在python中很容易求得一个矩阵的各行的和、平均数与列的和、平均数,例如:

\begin{lstlisting}[language=Python]
A = np.arange(1,13).reshape((3,4),order='F')
A
\end{lstlisting}

\begin{lstlisting}[language=Python]
A.sum(axis=1)#行和
\end{lstlisting}

\begin{lstlisting}[language=Python]
A.mean(axis=1)#行平均
\end{lstlisting}

\begin{lstlisting}[language=Python]
A.sum(axis=0)#列和
\end{lstlisting}

\begin{lstlisting}[language=Python]
A.mean(axis=0)#列平均
\end{lstlisting}

\hypertarget{xx}{%
\section{矩阵X'X的逆}\label{xx}}

在统计计算中,我们常常需要计算这样矩阵的逆,如OLS估计中求系数矩阵。R中的包``strucchange''提供了有效的计算方法。

\begin{lstlisting}[language=Python]
A = np.arange(1,13).reshape((3,4),order='F')
A
\end{lstlisting}

\begin{lstlisting}[language=Python]
A.sum(axis=1)#行和
\end{lstlisting}

\begin{lstlisting}[language=Python]
A.mean(axis=1)#行平均
\end{lstlisting}

\begin{lstlisting}[language=Python]
A.sum(axis=0)#列和
\end{lstlisting}

\begin{lstlisting}[language=Python]
A.mean(axis=0)#列平均
\end{lstlisting}


\end{document}
