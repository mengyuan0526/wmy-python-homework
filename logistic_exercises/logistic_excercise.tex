\documentclass[UTF8,a4paper,12pt]{ctexart}  % Latex 去掉上面的语句,加上本语句
\usepackage{xeCJK}
\setCJKmainfont[BoldFont=AdobeHeitiStd-Regular]{AdobeSongStd-Light}
\setCJKfamilyfont{song}{AdobeSongStd-Light}
\setCJKfamilyfont{hei}{AdobeHeitiStd-Regular}
\setCJKfamilyfont{kai}{AdobeKaitiStd-Regular}
\setCJKfamilyfont{fs}{Sun Yat-sen Hsingshu}
% \setCJKmainfont[BoldFont=SimHei]{SimSun}


\renewcommand{\contentsname}{\centerline{\textcolor{violet}{目 \ \ 录}}}    % 将Contents改为目录
\renewcommand{\abstractname}{摘 \ \ 要}      % 将Abstract改为摘要
\renewcommand{\refname}{参考文献}            % 将Reference改为参考文献
\renewcommand\tablename{表}
\renewcommand\figurename{图}
\renewcommand{\today}{\number\year 年 \number\month 月 \number\day 日}

\usepackage[dvipsnames]{xcolor}
\PassOptionsToPackage{colorlinks=true,citecolor=blue, urlcolor=blue, linkcolor=violet, bookmarksdepth=4}{hyperref}

\usepackage{lscape}
\usepackage{indentfirst}
\usepackage{textcomp}                      % provide many text symbols
\usepackage{setspace}                      % 各种间距设置

% ---------------------------------Table------------------------------
\usepackage{longtable}
\usepackage{booktabs}
\usepackage{array}                         % 提供表格中每一列的宽度及位置支持
\usepackage{rotating}
\usepackage{multirow}
\usepackage{wrapfig}
\usepackage{colortbl}
\usepackage{pdflscape}
\usepackage{tabu}
\usepackage{threeparttable}
\usepackage{threeparttablex}
\usepackage[normalem]{ulem}
\usepackage{makecell}
\newcolumntype{L}[1]{>{\raggedright\let\newline\\\arraybackslash\hspace{0pt}}m{#1}}
\newcolumntype{C}[1]{>{\centering\let\newline\\\arraybackslash\hspace{0pt}}m{#1}}
\newcolumntype{R}[1]{>{\raggedleft\let\newline\\\arraybackslash\hspace{0pt}}m{#1}}

%% 参考文献
\usepackage{gbt7714}
\usepackage{natbib}
\setlength{\bibsep}{0.5pt}


\usepackage[utf8]{inputenc}
% \usepackage[T1]{fontenc} % [T1] 主要支持东欧等国家重音符 , 与下面 consolas 冲突
\usepackage{fontenc}
\usepackage{fixltx2e}
\usepackage{graphicx}
\usepackage{float}
\usepackage{wrapfig}
\usepackage{soul}
\usepackage{textcomp}

\newcommand\hmmax{0} %% 防止Too many math alphabets used in version normal.
\newcommand\bmmax{0} %% 防止Too many math alphabets used in version normal.

\usepackage{lmodern,bm}   % 必需出现在amsmath等包前面,否则会出错
\usepackage{amsmath}
\usepackage{marvosym}
\usepackage{wasysym}
\usepackage{latexsym}
\usepackage{amssymb}
\usepackage{hyperref}
\usepackage{listings}
\usepackage{tikz}

\setmonofont{Consolas} % listings 中支持 consolas 字体,必需配合上面\usepackage{fontenc} 中不出现[T1]才可以

\lstset{numbers=left, numberstyle=\ttfamily\tiny\color{Gray}, stepnumber=1, numbersep=8pt,
  frame=leftline,
  framexleftmargin=0mm,
  rulecolor=\color{CadetBlue},
  backgroundcolor=\color{Periwinkle!20},
  stringstyle=\color{CadetBlue},
  flexiblecolumns=false,
  aboveskip=5pt,
  belowskip=0pt,
  language=R,
  basicstyle=\ttfamily\footnotesize,
  columns=flexible,
  keepspaces=true,
  breaklines=true,
  extendedchars=true,
  texcl=false,  % 必须设置为false设置为true的时候 R 代码中不能含有多个注释符号 #
  upquote=true,
  showstringspaces=false,
  keywordstyle=\bfseries,
  keywordstyle=\color{Purple},
  xleftmargin=20pt,
  xrightmargin=10pt,
  morecomment=[s]{\#}{\#},
  commentstyle=\color{OliveGreen!60}\scriptsize,
  tabsize=4}

\tolerance=1000

%======================== 根据选项设置代码处理方式 RMARKDOWN 中独有=============

\providecommand{\tightlist}{\setlength{\itemsep}{0pt}\setlength{\parskip}{0pt}}
\newcommand{\passthrough}[1]{\lstset{mathescape=false}#1\lstset{mathescape=true}}

\author{\CJKfamily{kai} 金 \enspace 林 \\ \CJKfamily{kai} 中南财经政法大学统计系 \\ jinlin82@gmail.com}
% ------------------------Chapter Section Title-------------------------
%--- 英文期刊标题 -----
\usepackage{titlesec}
\titleformat{\section}{\large\bfseries}{\thesection}{1em}{}
\titleformat{\subsection}{\normalsize\bfseries}{\thesubsection}{0.5em}{}
\titlespacing{\section}{0pt}{1ex plus 1ex minus .2ex}{1ex plus 1ex minus .2ex}
\titlespacing{\subsection}{0pt}{0.5ex plus 1ex minus .2ex}{0.5ex plus 1ex minus .2ex}
%--- 中文期刊标题 -----
% \CTEXsetup[name={,、}, number={\chinese{section}}, aftername={},
% format={\large \heiti }, indent={24pt},
% beforeskip={1ex plus 1ex minus .2ex},
% afterskip={1ex plus 1ex minus .2ex}]
% {section}
% \CTEXsetup[name={(,)}, number={\chinese{subsection}}, aftername={},
% format={\normalsize \bfseries \songti}, indent={\parindent},
% beforeskip={0.5ex plus 1ex minus .2ex},
% afterskip={0.5ex plus 1ex minus .2ex}]
% {subsection}
% \CTEXsetup[name={,.}, number={\arabic{subsubsection}},
% aftername={}, format={\normalsize \bfseries \songti},indent={\parindent},
% beforeskip={0ex plus 1ex minus .2ex},
% afterskip={0.2ex plus 1ex minus .2ex}]
% {subsubsection}

% ------------------------Figure and Table Caption---------------------
%\makeatletter                        % 图表标题格式设置
%\renewcommand{\fnum@table}[1]{\small \bfseries\textcolor{Violet}{\tablename\thetable~~}}
%\renewcommand{\fnum@figure}[1]{\small \CJKfamily{hei} \textcolor{Violet}{\figurename\thefigure~~}}
%\makeatother

\usepackage[skip=0pt, labelsep=quad, font={small, bf}, labelfont={color={Violet}}]{caption}

\renewcommand{\thefigure}{\arabic{figure}}
\renewcommand{\thetable}{\arabic{table}}
\newcommand{\HRule}{\rule{\linewidth}{0.5mm}}

\usepackage[top=2cm,bottom=2cm,left=3cm,right=3cm]{geometry}
\sloppy
\linespread{1.2}                    % 设置行距
\setlength{\parindent}{24pt}        % 段落缩进
\setlength{\parskip}{1ex plus 0.5ex minus 0.2ex}
\pagestyle {plain}                  % 去掉页眉
\setcounter{secnumdepth}{4}

%%% Change title format to be more compact
\usepackage{titling}

% Create subtitle command for use in maketitle
\newcommand{\subtitle}[1]{
  \posttitle{
    \begin{center}\large#1\end{center}
    }
}

\setlength{\droptitle}{-2em}
  \title{\LARGE\textbf{Logistic回归}}
  \pretitle{\vspace{\droptitle}\centering\huge}
  \posttitle{\par}
  \author{王梦圆}
  \preauthor{\centering\large\emph}
  \postauthor{\par}
  \predate{\centering\large\emph}
  \postdate{\par}
  \date{2020-02}



\begin{document}
\maketitle

\hypertarget{zeligturnoutraceageeducateincomevote0-110racewhite0others1.logistc}{%
\subsubsection{选择Zelig包里面的turnout数据集,这个数据集是为了确定投票率与被选举人的种族(race)、年龄(age)、受教育程度(educate)和收入(income)是否有关。因变量vote是0-1型变量,赞成为1,反对为0。种族(race)变量是一个分类型自变量,在进行回归分析时,可以将``white''记为0,``others''记为1.下面对该数据集进行二分类Logistc回归分析。}\label{zeligturnoutraceageeducateincomevote0-110racewhite0others1.logistc}}

\paragraph{数据准备和模型建立}

\begin{lstlisting}[language=Python]
import numpy as np
import pandas as pd
import statsmodels.api as sm 
import statsmodels.formula.api as smf
from sklearn.model_selection import train_test_split
df = sm.datasets.get_rdataset("turnout",package="Zelig",site="D:/github_repo/Rdatasets").data
\end{lstlisting}

\begin{lstlisting}[language=Python]
df

#对数据进行处理:去空值,处理分类变量race
\end{lstlisting}

\begin{lstlisting}[language=Python]
df.isnull().sum()#没有空值
\end{lstlisting}

\begin{lstlisting}[language=Python]
df['race']=df['race'].replace("white",0)
df['race']=df['race'].replace("others",1)
df['race'].unique()
\end{lstlisting}

\begin{lstlisting}[language=Python]
df.index=np.arange(df.shape[0])
X=df.iloc[:,:4]
X
\end{lstlisting}

\begin{lstlisting}[language=Python]
Y=df['vote']
p=Y.sum()/len(Y)#投票比率为0.746

#将数据集随机划分为训练子集和测试子集,并返回划分好的样本和标签
X_train,X_test,y_train,y_test=train_test_split(X,Y,test_size=0.2,random_state=0)
#x_train训练集特征值
#y_train训练集目标值
#x_test测试集特征值
#y_text测试集目标值,真实值

#拟合logistic回归方程
results=sm.Logit(y_train,X_train).fit()
\end{lstlisting}

\begin{lstlisting}[language=Python]
print(results.summary())
\end{lstlisting}

Logistic回归方程:
\[\log \frac{p}{1-p}=-0.5874 * \text { race }+0.0063 * \text { age }+0.0367 * \text { educate }+0.1559 * \text { income }
\]
根据输出的结果显示,在显著性水平为0.05下,四个变量的P值均小于0.05,即四个自变量种族、年龄、受教育程度和收入都对是否决定投票都有显著影响。

\paragraph{模型准确率}

\begin{lstlisting}[language=Python]
#预测数据
y_predict = results.predict(X_test)
y_predict
\end{lstlisting}

\begin{lstlisting}[language=Python]
y_predict = np.where(y_predict>0.5,1,0)
accuracy = (y_predict==y_test).sum()/len(y_test)
accuracy
\end{lstlisting}

以0.5作为阈值,预测准确率为0.74,即用种族、年龄、受教育程度和收入这四个变量估计投票的概率是74\%, 因为除了被选举人自身的优势之外,一些政治因素也是造成预测准确率不是很高的因素。

\hypertarget{kmsurvaidslogisticadultinfectinductinfectinduct}{%
\subsubsection{利用KMsurv包里面的aids数据集,利用Logistic回归分析法调查二分类变量adult与infect、induct的关系。infect和induct变量都为连续型变量。}\label{kmsurvaidslogisticadultinfectinductinfectinduct}}

\paragraph{数据准备和建模}

\begin{lstlisting}[language=Python]

dat = sm.datasets.get_rdataset("aids",package="KMsurv",site="D:/github_repo/Rdatasets").data
dat

#对数据进行处理:去空值,处理分类变量race
\end{lstlisting}

\begin{lstlisting}[language=Python]
dat.isnull().sum()#没有空值
\end{lstlisting}

\begin{lstlisting}[language=Python]
dat.index=np.arange(dat.shape[0])
X=dat.iloc[:,:2]
Y=dat['adult']
p=Y.sum()/len(Y)

#将数据集随机划分为训练子集和测试子集,并返回划分好的样本和标签
X_train,X_test,y_train,y_test=train_test_split(X,Y,test_size=0.2,random_state=0)

#拟合logistic回归方程
results=sm.Logit(y_train,X_train).fit()
\end{lstlisting}

\begin{lstlisting}[language=Python]
print(results.summary())
\end{lstlisting}

回归方程:
\[\log \frac{p}{1-p}=0.1037 * \text { infect }+0.7287 * \text { induct }\]
又因为infect变量的P值为0.103,在显著性水平0.05下,infect变量对因变量adult率影响不显著,因此,除去infect变量后再次进行Logistic回归分析.

\begin{lstlisting}[language=Python]
X_new=dat['induct']
X_train,X_test,y_train,y_test=train_test_split(X_new,Y,test_size=0.2,random_state=0)
results=sm.Logit(y_train,X_train).fit()
\end{lstlisting}

\begin{lstlisting}[language=Python]
print(results.summary())
\end{lstlisting}

回归方程:
\[\log \frac{p}{1-p}=0.9269 * \text { induct }\]
变量induct的P值为0,说明变量induct的影响是显著的。
\#\#\#\# 模型准确率预测

\begin{lstlisting}[language=Python]
predict = results.predict(X_test)  
y_predict2 = np.where(predict>0.5,1,0)
accuracy2=(y_test==y_predict2).sum()/len(y_test)
accuracy2#0.847
\end{lstlisting}


\end{document}
